\documentclass{beamer}
\mode<presentation>{\usetheme{CambridgeUS}}
\usecolortheme{beaver}
\usepackage{caption}
\beamertemplatenavigationsymbolsempty
\usepackage{fancybox}
\usepackage{csquotes}


\newenvironment{myfont}{\fontfamily{LinuxLibertineT-OsF}\selectfont}{\par}
\usepackage[url=false,
            isbn=false,
            backref=false,
            style=ieee,
            backend=biber,
            abbreviate=true,
            natbib=true]{biblatex}
           
%\usepackage[hyperfootnotes=false]{hyperref} 
\addbibresource{presbib.bib}
\bibliography{}

\usepackage[normalem]{ulem}
\setbeamertemplate{bibliography item}[text]


\renewcommand{\bibfont}{\footnotesize}

%\ExecuteBibliographyOptions{labelyear}

\DeclareCiteCommand{\morecite}[\mkbibparens]%
  {\usebibmacro{prenote}}%
  {\usebibmacro{citeindex}%
   \printtext[bibhyperref]{%
     \printfield{prefixnumber}%
     \printfield{labelnumber}%
     \ifbool{bbx:subentry}
       {\printfield{entrysetcount}}
       {}}%
   \ifnameundef{labelname}%
    {}
    {\setunit{\addcomma\space}%
     \printnames{labelname}}%
   \setunit{\addcomma\space}%
   \printfield{labelyear}}
  {\multicitedelim}%
  {\usebibmacro{postnote}}%

\usepackage{pst-node}
\psset{arrowscale=2,arrows=->}
\def\VPh{\vphantom{\displaystyle\sum_{i=n}^m {i^2}}}
\def\psBox#1#2{\psframebox[fillcolor=#1,fillstyle=solid]{\VPh\displaystyle#2}}

\title[Analysis of CS Theory Assignments]{Towards Automated Analysis of CS Theory Assignments}
\institute[RIT]{Rochester Institute of Technologoy}
\date{\today}


\begin{document}
\maketitle

\section{The Problem}

\begin{frame}
\frametitle{Tools for CS theory students}

\begin{itemize}
\item Computer science theory courses introduce automata and languages
\vspace{0.1in}
\item Software for analyzing, grading, and providing feedback on automata and grammars would benefit students
\vspace{0.1in}
\item Much work has already been done for regular languages \footfullcite{Bilska97} \footfullcite {Alur:2013} 
\vspace{0.1in}
\item But many of the questions we'd like to answer about context-free languages are undecidable
\end{itemize}
\end{frame}

\begin{frame}
\frametitle{Ambiguity and equivalence}
\begin{block}{Ambiguity}
Given a grammar $G$ is there a word $w , \; w \in \mathcal{L}(G)$, that has more than one leftmost derivation?
\end{block}

\pause 
\vspace{0.2in}
$G:$\\
\hspace*{10pt} $S \rightarrow  A \; | \; \epsilon$\\
\hspace*{10pt} $A \rightarrow   \epsilon$
\end{frame}

\begin{frame}
\frametitle{Ambiguity and equivalence}
\begin{block}{Equivalence}
Given grammars $G$ and $G'$, does $\mathcal{L}(G) = \mathcal{L}(G')$? 
\end{block}

\pause 
\vspace{0.2in}
$G:$\\
\hspace*{10pt} $S \rightarrow  A \; | \; \epsilon$\\
\hspace*{10pt} $A \rightarrow   \epsilon$

\vspace{0.2in}
$G':$\\
\hspace*{10pt} $S \rightarrow \epsilon$
\end{frame}

\begin{frame}
\frametitle{Because these problems are undecidable ...}
\begin{itemize}
\item Can't automate grading
\vspace{0.1in}
\item Can't automate feedback
\end{itemize}
\end{frame}

%-----------------------------------------------------
\section{The Approach}

\begin{frame}
\frametitle {Bounded versions of problems}

\begin{block}{Ambiguity}
Given a grammar $G$, is there a word $w, \; w \in \mathcal{L}(G) , \; |w| \leq k$ that has more than one leftmost derivation?
\end{block}

\vspace{0.2in}

\begin{block}{Equivalence}
Given grammars $G$ and $G'$, does $\mathcal{L}(G) = \mathcal{L}(G')$ for all words of length $\leq k$? 
\end{block}

\end{frame}


\begin{frame}
\frametitle {CFGAnalyzer}
\begin{itemize}
\item Axelsson, Heljanko, and Lange introduced an algorithm and a tool for converting bounded instances of problems on context-free grammars into boolean formulae in conjunctive normal form \footfullcite{CFGAnalyzer}

\vspace{0.1in}
\end{itemize}

\only<1-1>{
		\hspace{10pt}$G: $\\
		\hspace{20pt} $S \rightarrow 0 S 0 \; | \; 1 S 1 \; | \; \epsilon  \; \; \; \; \implies \; \; \; \; (a_0 \vee b_0 \vee c_1) \wedge (a_2 \vee \neg b_1 \vee \neg d_0 ) \wedge ... $
}

\only<2-2>
{\begin{itemize} 
\item Once the problem is expressed in CNF, a SAT solver can find satisfying assignments (if any exist)! 
\vspace{0.1in}
\item A satisfying assignment would encode a witness for ambiguity or inequivalence
\end{itemize}
\centering $w | w \in \mathcal{L}(G), w \not \in \mathcal{L}(G')$}

\only<3-3>
{\begin{itemize}
\item A modification to this algorithm would find all witnesses, giving a student a sense of how wrong their solution might be or how ambiguous their grammar is 
\end{itemize}
\centering $W =\{w | w \in \mathcal{L}(G), w \not \in \mathcal{L}(G')$\}}

\end{frame}

%------------------------------------------------------
\section{Results}

\begin{frame}
\frametitle{What is a reasonable value for the bound?}
\begin{figure}[H]
	\centering
	\begin{minipage}{0.49\textwidth}
		\centering
		\includegraphics[scale=0.30]{firstwitnessambig}
	\end{minipage}
	\begin{minipage}{0.49\textwidth}
		\centering
		\includegraphics[scale=0.30]{firstwitnessequiv}
	\end{minipage}
	\caption*{What proportion of the benchmark data has a shortest witness of length $n$?}
\end{figure}
\end{frame}


\begin{frame}
\frametitle{Integration of PicoSAT}
\begin{figure}[H]
	\centering
	\begin{minipage}{0.49\textwidth}
		\centering
		\includegraphics[scale=0.30]{zchaffambig}
	\end{minipage}
	\begin{minipage}{0.49\textwidth}
		\centering
		\includegraphics[scale=0.30]{picosatambig}
	\end{minipage}
	\caption*{Execution times for grammars with witnesses of length $k$ in zChaff \footfullcite{chaff} vs. PicoSAT \footfullcite{picoSAT}}
\end{figure}

\end{frame}

\begin{frame}
\frametitle{Future work}
\begin{itemize}
\item Integrate the CFGAnalyzer tool, along with its modifications, into JFLAP
\vspace{0.1in}
\item Explore additional methods of assigning partial credit (edit distance?)
\vspace{0.1in}
\item Test on more student samples!
\end{itemize}
\end{frame}

\section{References}
\begin{frame}[t,allowframebreaks] 
	\nocite{*}
	\frametitle{References}
	\printbibliography
\end{frame}

\section{Conclusion}
\begin{frame}
\frametitle{Thank you!}
\begin{center} \LARGE Questions?
\end{center}
\end{frame}
\end{document}